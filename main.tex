\documentclass{beamer}
\usetheme{CambridgeUS}
\usepackage{amsmath}

\def\inputGnumericTable{}
\setbeamertemplate{caption}[numbered]{}

\usepackage{enumitem}
\usepackage{amsmath}
\usepackage{amssymb}
\usepackage{gensymb}
\usepackage{graphicx}
\usepackage{txfonts}
\usepackage[latin1]{inputenc}
\usepackage{color}
\usepackage{array}
\usepackage{longtable}
\usepackage{calc}
\usepackage{multirow}
\usepackage{hhline}
\usepackage{ifthen}

\usepackage{lscape}




\title{Assignment 4 Probability}
\author{Narsupalli Sai Vamsi}
\date{June 2022}


\begin{document}
\begin{frame}
\titlepage
 \begin{abstract}
     This pdf consists the solution to the question 6.46 from in Papoulis pillai
 \end{abstract}   
\end{frame}
\begin{frame}{Outline}
\tableofcontents
\end{frame}
\section{Question 6.46}
\begin{frame}{Question 6.46}
 (Q6.46) Let" and y be independent Poisson random variables with parameters $\lambda_1$ and $\lambda_2$ respectively. Show that the conditional density function of x given x + y is binomial 
\end{frame}
\section{Solution}
\begin{frame}{Solution}
\begin{block}{Solution}
    The moment generating function of X and Y are given by\\
  $$
  \Gamma_x{(z)} = e^{\lambda_1(z-1)}
  $$
  $$
  \Gamma_y{(z)} = e^{\lambda_1(z-1)}
  $$
  And\\
  $$
  \Gamma_{x+y}{(z)} = e^{(\lambda_1+\lambda_2)(z-1)}
  $$
  $$
  Z\sim P(\lambda_1+\lambda_2)
  $$ 
\end{block}
\end{frame}
\section{Solution}
\begin{frame}{Solution}
\begin{block}{Solution}
 we know that in poissons distribution\\
 $$P(X+Y=k) = e^{-(\lambda_1+\lambda_2)} \frac{(\lambda_1+\lambda_2)^k}{k!} $$\\
 and what we need to find is the conditional function so lets take some k \\
 $$P(X=k|X+Y=n) = \frac{P(X=k,X+Y=n)}{P(X+Y=n)}
 = \frac{P(X=k)P(Y=n-k)}{P(X+Y=n)}
 $$
 $$
  =\frac{e^{-\lambda_1}(\lambda_1^k/k!)e^{-\lambda_2}(\lambda_2^{n-k}/(n-k)!)}{e^{-(\lambda_1+\lambda_2)}(\lambda_1+\lambda_2)^n/n!} 
 $$
\end{block}
\end{frame}
\section{Solution}
\begin{frame}{Solution}
\begin{block}{Solution}
Here we took k which runs from 0 to n and we need to prove that the distribution is binomial.
$$
=
\begin{pmatrix}
n\\
k
\end{pmatrix}
\Big(\frac{\lambda_1}{\lambda_1+\lambda_2}\Big)^k \Big(\frac{\lambda_2}{\lambda_1+\lambda_2}\Big)^{n-k}
$$
$$
= 
\begin{pmatrix}
n\\
k
\end{pmatrix}
m^k(1-m)^{n-k}
$$
Binomial (n,m) where m
$
 = \frac{\lambda_1}{\lambda_1+\lambda_2}
$
\end{block}
\end{frame}
\end{document}
